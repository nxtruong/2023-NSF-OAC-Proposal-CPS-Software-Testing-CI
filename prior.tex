\section{Results From Prior NSF Support}

\paragraph{PI Groce:}
The most relevant prior NSF support for PI Groce is
CCF-1217824, ``Diversity and Feedback in Random Testing for Systems
Software,'' with a total budget of \$491,280 from 9/2012 until 9/2015,
a collaborative proposal with John Regehr of the University of
Utah. {\bf Intellectual Merit:} The results of CCF-1217824 included a
number of advances to practical automated generation and use of tests,
a key focus of this proposal as well.  E.g., an approach to creating ``quick
tests'' from a test suite by minimizing each
test with respect to its code coverage \cite{icst2014}, won the
Best Paper award at the 2014 International Conference on Software
Testing.  CCF-1217824
produced a general set of results focused on making automated random
testing usable by practitioners, and using symbolic execution on
larger, realistic software.  Publications resulting from this grant
were numerous
\cite{Onward14,PLDI13,issta14,icst2014,helphelp,DirectedSwarm,stvrcausereduce,tstlsttt,ISSTA15,ASEAdeq}. {\bf
  Broader
  Impact:} The results of CCF-1217824 have been used in teaching software
engineering classes.
Work from the project contributed to the discovery of previously
unknown faults in important software
systems, including LLVM and GCC, and is widely used in compiler
testing~\cite{ZhendongPLDI14,beginnerluck,dewey2015fuzzing,le2015randomized}.
Tools and data sets from CCF-1217824 are available via GitHub in
multiple repositories and projects (TSTL, Csmith, CReduce, etc.).

\paragraph{Co-PI Flikkema:} Flikkema is co-PI on the Southwest Experimental Garden Array (SEGA)
funded by an NSF development MRI
(DEB-1126840), with a total budget of \$2,848,876 from 10/2011 until
9/2017. {\bf Intellectual Merit:} SEGA is a facility
distributed across a 1615m elevation gradient in Arizona that supports
long-term research to increase understanding of and mitigate climate
change using knowledge of genetic variation in species of concern. It
consists an array of eleven gardens and supporting distributed
monitoring and control cyberinfrastructure for the study of
gene-by-environment interactions and enabling development of
strategies to best manage for future climates. {\bf Broader
  Impact:}  With 9 successfully completed projects to date, SEGA
currently supports 11 experiments and has resulted in over 35
publications and 20 conference presentations.  SEGA results are
available online~\cite{SEGA}.

\paragraph{Co-PI Nghiem:} has not received any NSF support.
