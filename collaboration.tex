%%%%%%%%%%%%%%%%%%%%%%%%%%%%%%%%%%%%%%%%%%%%%%%%%%%%%%%%%%%%%
\documentclass{article}

\begin{document}

\begin{center}
{\Large\sf\textbf{Formal Methods in the Field: A Pathway for Combining Formal, Static, and Dynamic Analysis of Real-World Embedded Systems: Collaboration Plan}}
\end{center}

The PIs are all faculty in the School of Informatics, Computing and Cyber Systems at Northern Arizona University.  They all work in the same building, which also hosts the SEGA lab (as opposed to the field installations which are located in various places, some near Northern Arizona University, others more remotely).  PIs will meet weekly for technical discussion and project status updates, depending on project status (and already do so, in discussions of the preliminary work) to coordinate their efforts, and ensure that tasks in the work plan are proceeding correctly, and to receive feedback on current status of tools.  These meetings will alternate between meetings with just PIs, focused on higher level decision-making, and meetings including all involved students, focusing on status of individual work packages.
In addition to weekly project meetings, there will be special out-of-band meetings to demonstrate significant new functionalities in tools, or to focus particularly on application of functionalities to specific SEGA or robotics components.

The existing SEGA implementation code and fault(s) form a core concern that helps focus PI interactions, and enables easier communication of technical results between static and dynamic analysis experts and experts in the embedded systems domain.  PIs will also set up a project repository, separate from DeepState, SEGA, or robotics code repositories (already in existence) and use PR, tagged Issues, and other GitHub-supported collaborative software development best practices to ensure documentation of development, and team awareness of code status for components.  Travis CI or GitHub Actions-based testing will be used to ensure all project members are aware of build or correctness problems with project code.  Graduate students, undergraduate students, and PIs will also share a Project Slack Instance, linked to the code repo, to make discussion of issues arising during the project even easier, outside of scheduled meeting times.

To summarize, collaboration will be coordinated through multiple overlapping methods, to ensure success:
\begin{itemize}
\item Weekly project meetings (alternating between PIs and PIs + students)
\item Meetings of PIs with involved students
\item Demonstration and SEGA-focused meetings
\item Demonstration and robotics-focused meetings
\item GitHub repositories for SEGA, robotics, and DeepState extensions
\item Project GitHub repository
\item GitHub PRs and Issues to coordinate development
\item CI (Travis or GitHub actions) for code status awareness
\item Project Slack linked to GitHub repo and CI, for effective communication
\end{itemize}

\end{document}
