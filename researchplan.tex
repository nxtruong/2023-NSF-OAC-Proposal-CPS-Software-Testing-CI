\section{Research Plan}

%\subsection{From Timed Automata to \framac}
%\label{sec:ta2framac}
%\input{automatatoframac}

\subsection{DeepState and Automated Test Generation}
\label{sec:framac2deepstate}
While \framac allows, ideally, for proof of correctness of annotated code, in many real-world instances it will be impossible to prove correctness, either because the proof is too hard to construct or because the code is not in fact correct.  While \framac provides some mechanisms for generating possible counterexamples to a proof, and limited test case generation, it is far from ideal in this setting.  A full workflow for verification of realistic systems, therefore, requires a first-class \emph{dynamic} analysis component.  Furthermore, such a component should not limit itself to a single method for generating test cases, as with tools such as KLEE~\cite{KLEE}, Pex~\cite{Pex}, or the test generation tools provided by \framac~\cite{PathCrawler}.  Predicting which test generation methods will discover a fault in code, or simply scale to provide effective exploration of code paths, is notoriously difficult.  Furthermore, most tools are research prototypes and have known bugs that prevent application to some subset of programs.  We therefore aim to provide a \emph{flexible} dynamic front-end that provides a one-stop solution to the problem of dynamic analysis, either to provide confidence in code that cannot be proven correct, or to discover a counterexample showing that the code is not correct.  \deepstate~\cite{DeepState} provides such a front-end.

The research challenge is to translate \acsl-annotated code for use in \framac into a full-featured \deepstate test harness.  This problem can further be broken down into four key-subproblems:

\begin{enumerate}[labelsep=3pt,leftmargin=12pt]
\item The \emph{specification} of correctness must be translated into an executable form.  To some extent, the existence of the \eacsl executable subset of \acsl, and libraries for runtime checking of properties satisfies this condition.  DeepState can support any C/C++ executable method of checking for correctness.  However, because DeepState provides more back-ends, some executable specifications may need to be modified to be efficiently handled when the DeepState back-end is a symbolic execution tool.  Additionally, it is important to instrument the executable code that support specifications in order to make the coverage of the specification itself visible to fuzzer back-ends, such as libFuzzer.  Moreover, DeepState's nature as a test generation tool means that it support constructs, such as ForAll, Minimum, and Maximum, that are not normally available in executable specifications.  Tailoring \eacsl usage for DeepState therefore requires a custom effort, including extending the semantics of ``executable'' specifications and optimizing the implementation for symbolic execution and fuzzing.  Finally, because our domain critically involves timing, we need to implement DeepState handling (and \eacsl representations for) deadlines, and specification of function-level deadlines including arbitrary, specified, ``runtimes'' for code that operates via simulation rather than real hardware, including nondeterministic expression of the timing constraints on calls, and introducing ghost branches that make timing impacts visible to coverage-driven fuzzers.
\item The \emph{assumptions} that control which tests are considered valid must be translated in the same way; normally, \eacsl simply translates these into further assertions (as pre-conditions to check at runtime), but in DeepState, we need to distinguish between {\tt ASSUME} constructs where failure indicates an invalid test and {\tt ASSERT} constructs where failure indicates a failing test.  Additionally, the same optimizations and visibility-to-fuzzer improvements as for the specification must be provided.
\item The inputs to a function must be translated into code controlling the input values that DeepState must generate in a test, including ranges and types.  When input types are simple, this process is straightforward; however, when functions take, e.g., arbitrarily sized arrays or linked lists, or other complex structures, this becomes a problem of constructing a test harness that (1) makes fuzzing and symbolic execution scalable but (2) allows large enough structures to expose subtle bugs.  Moreover, because DeepState supports strategies for input generation, such as forking concrete states for values too complex for symbolic execution using the {\tt Pump} construct, the translation must determine when such strategies are appropriate.
\item In many cases, checking a single function may not be an effective way to detect faults; only a sequence of API calls can expose a problem in a system (e.g., that a function produce a state that causes another function to violate an invariant).  \acsl annotations provide enough information for a fully-automated translation to a harness enabling dynamic analysis in the case of proving properties of a single function, but this is no longer true for groups of functions.  Moreover, even in cases where the violation of a specification can, in theory, be discovered without calling multiple functions, the state space described by the precondition for a function may be too large to explore with a fuzzer or symbolic execution tool.  In such cases, exploring the space described by valid calls of other functions has two benefits:  first, the space described by a sequence of calls may be much smaller, and easier to explore, than the full set of possible input values to a function.  Second, errors in this portion of the input space are more clearly realistic scenarios.  Even if a precondition is not sufficiently restrictive to guarantee correct behavior, if the ``bad'' inputs are never, in practice, generated by the functions that modify system state, the fault may never appear in practice.  In cases where constructing a sufficiently exact precondition is difficult for engineers, such ``in-use'' verification may be the only avenue to system assurance; proof is impossible without a restrictive enough precondition, and dynamic methods may scale very poorly to, e.g., a large unstructured byte buffer such as a hash table.  We propose to let users annotate (in an extension of \acsl) sets of functions to be tested as an API-call-sequence group.  E.g., annotating a set of file system functions ({\tt mkdir}, {\tt rmdir}, {\tt readdir}, etc.) as such a group could allow the automatic generation of a DeepState harness that checks for cases where a sequence of valid function calls can violate a precondition or cause a fault despite preconditions being satisfied.
\end{enumerate}

\begin{figure}[t]
  {\scriptsize
  \begin{code}

void update\_state(struct state\_t *state, uint64\_t bitvector) \{
  ASSUME(valid\_state(state));
  ASSUME(valid\_bv(bitvector));
  ...
\}

void process\_both\_sensor\_readings(struct state\_t *state) \{
  ASSUME(valid\_state(state)); 
  unit64\_t s1\_bv = acquire\_s1(); 
  update\_state(state, s1\_bv); 
  unit64\_t s2\_bv = acquire\_s2(); 
  update\_state(state, s2\_bv);  
\}
  
void process\_one\_sensor\_reading(struct state\_t *state) \{
  ASSUME(valid\_state(state)); 
  unit64\_t s1\_bv = acquire\_s1(); 
  update\_state(state, s1\_bv); 
\}

TEST(SensorReading, UpdateNeverSlow) \{
  struct state\_t state;
  DeepState\_SymbolizeData(\& state, \& state + sizeof(struct state\_t));
  uint64\_t bv = DeepState\_UInt64();
  DeepState\_Timeout([\&] \{update\_state(\&state, bv); \}, MAX\_EXPECTED\_UPDATE\_TIME);
\}

TEST(SensorReading, AvoidCrashes) \{
  struct state\_t state;
  DeepState\_SymbolizeData(\& state, \& state + sizeof(struct state\_t));
  for(int i = 0; i < TEST\_LENGTH; i++) \{
      OneOf([\&] {process\_both\_sensor\_readings(state);},
        [\&] {process\_one\_sensor\_reading(state);});
  \}
\} 
\end{code}
}
  \caption{Portion of a DeepState harness for sensor-reading code}
  \label{fig:assumption}
  \end{figure}

  These goals require significant advances in two areas of dynamic analysis: first, a complete and principled approach to the problem of handling pre-conditions/assumption semantics, and second, an investigation of how to let fuzzers take advantage of the much greater structure involved in property-based testing than in traditional unstructured fuzzing; this includes specification structure, so is inherently tied to the first problem.  Consider the code in Figure~\ref{fig:assumption}.  This defines two different DeepState tests of (largely omitted) software that reads sensor values and incorporates them into a system state structure.  The two tests check two different properties:  {\tt UpdateNeverSlow} ensures that no matter what valid inputs are given to it, updating the sensor is never too slow.  The property is checked, potentially, over \emph{all} valid inputs, not just ones produced by the actual sensor reading code in {\tt acquire\_s1} and {\tt acquire\_s2}.  The second, independent, test, {\tt AvoidCrashes} simply starts the system up in some valid state, and repeatedly either reads both sensors or reads only sensor one.  There is no explicit property here, only the expectation that the system will not crash; tests can be executed using, e.g. LLVM sanitizers to further check for integer overflow, other undefined behavior, and so forth.  Generating such harnesses automatically from \acsl specifications is a significant challenge, but our research agenda also includes solving problems that would appear even if the harness were generated by hand.  First, what is the proper semantics of the {\tt ASSUME} in {\tt update\_state}?  It depends on the test.  In {\tt UpdateNeverSlow}, a fuzzer will often generate an input value that violates the (possibly complex) requirements on valid states and sensor readings.  These invalid inputs should not be flagged as bugs (the default behavior of \eacsl), but instead the test should be abandoned but without indicating that it failed to the fuzzer.  However, in {\tt AvoidCrashes}, since we are not directly generating state values, that is, {\tt update\_state} is not an \emph{entry point} for the test, assumption violations should result in a failed test.  We aim to synthesize code to make assumptions automatically take on the proper semantics during test execution, without the user having to redefine the behavior.  The solution must also encode the difference between a path to search for and a constraint to satisfy for symbolic execution of tests.

  This point about preconditions/{\tt ASSUME} brings up a second point.  Preconditions, when they have an {\tt ASSUME} semantics, are fundamentally different than other branches in code, with respect to fuzzing.  By default, a fuzzer will attempt to explore the behavior of branches in {\tt valid\_state} and {\tt valid\_bv} just as it explores branches in {\tt update\_state} or the {\tt acquire} functions.  However, from the point of view of testing, this is not ideal.  It is often possible to enumerate a vast number of input paths that only define invalid inputs, and so produce very little real testing despite a major computational effort.  A classic example is ``testing'' a file system by producing a huge variety of unmountable file system images, rather than actually executing any POSIX operations at all~\cite{CFV08,AMAI}.  DeepState ``knows'' which branches are pre-conditions, and so can help a fuzzer avoid this problem.  In some fuzzers, this means prioritizing inputs to mutate based on whether they execute any code other than validity checks; but in ``smart fuzzers'' such as Angora~\cite{angora} and Eclipser~\cite{eclipser} that perform lightweight constraint-solving to cover branches, the process can be even more sophisticated.  We have begun discussions with the Eclipser team, and they confirm that identifying precondition code and devising suitable heuristics to handle it (e.g., never solve for a negation of a validity check) should improve Eclipser's performance in property-based fuzzing.  Devising effective heuristics to tune both ``classic'' mutation fuzzers and concolic gray-box fuzzers promises to improve test generation not only in the context of our workflow, but in general.

This effort also connects to a second fuzzing research thrust: making specification elements that do not correspond to simple code coverage visible to a fuzzer.  In this example, consider the {\tt DeepState\_Timeout} check (note that this itself is functionality we will develop as part of handing timing constraints in \framac and DeepState).  Unless we break down the timing analysis explicitly, coverage-driven fuzzers cannot distinguish an execution that is very slow (close to violating the constraint) from one that has the minimum execution time possible.  We propose to make timing of such specified events visible to a fuzzer, by modifying coverage bitvectors to incorporate bucketing of execution time.  Once we add such novel coverage measures, and introduce distinctions between coverage classes (as with preconditions), we will research how to balance competing priorities in more complex notions of coverage.  We assume that as we investigate real world examples, further challenging research problems in property-based fuzzing will arise and require improving fuzzer science and art.

% Additionally, in some cases, checking a single function may not be an effective way to detect faults; only a sequence of API calls can expose a problem in a system (e.g., that a function produce a state that causes another function to violate an invariant).  \acsl annotations provide enough information for a fully-automated translation to a harness enabling dynamic analysis in the case of proving properties of a single function, but this is no longer true for groups of functions.  Moreover, even in cases where the violation of a specification can, in theory, be discovered without calling multiple functions, the state space described by the precondition for a function may be too large to explore with a fuzzer or symbolic execution tool.  In such cases, exploring the space described by valid calls of other functions has two benefits:  first, the space described by a sequence of calls may be much smaller, and easier to explore, than the full set of possible input values to a function.  Second, errors in this portion of the input space are more clearly realistic scenarios.  Even if a precondition is not sufficiently restrictive to guarantee correct behavior, if the ``bad'' inputs are never, in practice, generated by the functions that modify system state, the fault may never appear in practice.  In cases where constructing a sufficiently exact precondition is difficult for engineers, such ``in-use'' verification may be the only avenue to system assurance; proof is impossible without a restrictive enough precondition, and dynamic methods may scale very poorly to, e.g., a large unstructured byte buffer such as a hash table.

% In principle, of course, users can write a new function (a kind of ``ghost function'' not really executed---in practice, a test harness) expressing the desired mix of API calls that preserve an invariant; however, this is a serious burden on a user, and users are likely to make errors in this task~\cite{CFV08,AMAI,scriptstospecs,groce2015verified,groce2018verified}; we instead propose to let users annotate (in an extension of \acsl) sets of functions to be tested as an API-call-sequence group.  E.g., annotating a set of file system functions ({\tt mkdir}, {\tt rmdir}, {\tt readdir}, etc.) as such a group could allow the automatic generation of a DeepState harness that checks for cases where a sequence of valid function calls can violate a precondition or cause a fault despite preconditions being satisfied.



\subsection{Other DeepState Extensions}

\paragraph{Bounded Model Checking:}
While automated test generation by fuzzing or binary-level symbolic execution can be highly effective as a means for finding bugs in code, other approaches are also needed to handle the kinds of code especially common in embedded contexts.  In particular, embedded software often includes a large number of functions that perform complex low-level bit operations, especially for interacting with hardware and ``parsing'' network packets (from traditional wireless or RF-derived signals) communicating in very low-level protocols.

CBMC, the C Bounded Model Checker~\cite{cbmcp} is a well-known tool that analyzes C programs using a translation to SAT or SMT queries based on a bounded unrolling of loops. CBMC is an actively developed project, and has been used extensively in real-world development for years, including in automotive/embedded code development at Bosch and General Electric~\cite{tiemeyer2019crest}, in analysis of Amazon Web Services infrastructure~\cite{awsmodel}, and in the analysis of flight software systems at NASA's Jet Propulsion Laboratory~\cite{CFV08}.  Using CBMC requires writing custom test harnesses using CBMC's API for expressing nondeterminism, and running the tool with a specified bound on loop executions, in addition to a number of other configuration options, which are often non-obvious (for example, which safety properties of code are checked by default has changed with some releases of the tool, sometimes causing silent failures of ``working'' CBMC harness code).

We propose to allow CBMC to be used as a backend for verification by DeepState, similarly to how DeepState supports symbolic analysis engines such as angr and Manticore.  It is notoriously hard to guess when SAT/SMT based approached to code analysis will work well and when they will fail to scale; using a DeepState harness will allow users to try both CBMC and other test generation approaches without the effort of writing multiple harnesses.  

\paragraph{Explicit-State Model Checking:}
\input{spinplan}

\paragraph{Timed Automata Model Skeleton Generation:}
As noted above, one of our core assumptions is that timed automata can
model the underlying protocols in many embedded systems.  However,
writing timed automata models using \uppaal~\cite{uppaal} and
\prism~\cite{KNP2011:CAV} is at present a skill only a small number of
embedded engineers have mastered.  In order to encourage more
engineers to make use of these powerful formalisms, we propose 1) to
enable DeepState to generate \emph{traces} of the annotations related
to timing that are covered during a run and 2) to build a tool to
combine and reconcile these traces into a skeleton model for \uppaal~\cite{uppaal} or
\prism~\cite{KNP2011:CAV} (as has been done to some extent for Java~\cite{liva2017extracting}).  The structure of code (function locations
of DeepState annotations) will be used to form the structure of the
model.  Additional annotations for, e.g. probabilities, may need to be
added if not present in the code annotations, though DeepState already
has a primitive support for expressing probabilities that we plan to extend.

\paragraph{Other Tools:}  Galois Inc. has expressed strong interest in
an (unpaid) ongoing research collaboration to integrate their C and
C++ relevant tools into DeepState as well, to support our general vision.

\subsection{Case Studies} % for Ecological Monitoring and Control}
\label{sec:case-study}

The above briefly introduces a number of problems that we know in advance must
be dealt with in order to enable a pathway for combining formal,
static, and dynamic analysis.    At heart, however, we aim to allow
\emph{case studies} to prioritize our efforts, and
are certain that other challenges will arise during these efforts.
The studies informing this research is the embedded software of wireless sensor nodes used in the Southwest Experimental Garden Array (SEGA)~\cite{ClaEtAl11,GhoEtAl2014,BelEtAl2015} and of wireless sensor nodes and mobile robots in the Distributed Sensing \& Computing Over Sparse Environments (DISCOVER) Platform.

%%%%%%%%%%%%%%%%%%%% Intro to SEGA, especially its problem and scale / complexity

\paragraph{Overview:}

The first case study informing this research is the embedded software used in the Southwest Experimental Garden Array (SEGA)~\cite{ClaEtAl11,GhoEtAl2014,BelEtAl2015}.
SEGA is a large collection of operational wireless sensor/actuator networks for monitoring and control of ecological systems, located at 17 sites in the states of Arizona and California.
Currently, SEGA consists of 138 wireless nodes and is planned to expand to a total of 154 nodes at 21 sites in the coming years.
As a genetics-based climate change research platform, SEGA allows scientists to quantify the ecological and evolutionary responses of species to changing climate conditions.
Multiple long-term and large-scale scientific experiments are conducted at SEGA sites.


The SEGA nodes use a multi-processor architecture.
A central processor provides services, including scheduling and dispatch of tasks, storage, and a message-passing interface for wireless networking.
Plug-in satellite processors handle transducer sampling, actuation, and related computational tasks.
In addition to allowing true parallelism, this architecture enables hardware-level improvements in energy efficiency, since each satellite can be optimized for its specific task.
More practically, it admits the rapid implementation of highly heterogenous nodes that incorporate a wide range of sensing and actuation capabilities.
% The current implementation of the architecture emphasizes energy efficiency~\cite{FliSENSORS2010,FliICC2011}.
% For example, all satellite processors are power-gated via central processor control; ensuring that satellite processors are depowered prevents satellite sleep-mode energy leakage.
% The power subsystem provides multiple power buses at different voltages, including an optically-isolated high-power bus for actuation.
% A variety of energy supplies are also supported, including battery-backed photovoltaic sources~\cite{FliEtAl12,KnaFli17}.
%
The nodes synchronously interact with neighbors in a multi-hop, self-organizing/healing network; synchronization is implemented as scheduled rendezvous in time slots; slot boundaries are managed by a lightweight global time synchronization protocol that is integrated with low-level communication synchronization.
The nodes use a custom time-triggered RTOS tightly integrated with a time/frequency-hopped PHY/MAC protocol.
This approach %, implemented using a time-triggered architecture on a custom RTOS,
minimizes communication energy cost, which dominates the overall energy consumption.

\paragraph{Problem:}

Because timing is critical and is determined by the embedded system hardware and software, most testing has occurred at the network level, with extensive in-lab testing with small networks and instrumented field tests.
However, it has been found in long-term deployments % (at dozens of field sites over years of operation)
that occasionally the networking fails and nodes become isolated---we think due to a complex set of subtle bugs rooted in different levels of timing abstraction.
When such a failure occurs, it often spreads from one node to others,
causing nodes to seek to rejoin and expend high levels of energy for radio operation and eventually deplete their energy sources.
Eventually, subnets, or sometimes the entire site, are disabled and humans must visit the site to reboot it.
Such failures could %cause scientific data to be lost or invalidated and, even worse, could
affect %damage
or even destroy (e.g., via over-watering), long-running scientific experiments.


Since access to SEGA installations can be difficult, and in the long run many may be located so remotely that it is cost-prohibitive to send humans to address problems, discovering the source of these in-operation faults, identifying other faults, and generally improving the reliability of the system is critical.
We therefore aim to use SEGA (in particular the protocol in question and its implementation) as our primary case study.
This will enable us to apply our approach in a practical setting, and ensure that what we produce is actually usable by engineers of real systems.
%
SEGA is an ideal case study for several reasons.
First, the abovementioned network problem enables exploring how to design, prove, and test time-critical systems in a way that does no harm: human life is not affected in this application, and data is not lost since all sensed information is logged as a local back-up.
On the other hand, reliable operation is important, and failure costly. %, because access to manually fix problems can be problematic even with current installations, and in the future this problem will only grow.
Finally, this application uses common data structures for task control blocks, and the operating system at each node schedules and dispatches both periodic and pseudo-randomly scheduled tasks.
Thus the system is representative of %a good example of the
general applications of scheduling and synchronization in time-critical systems.
The embedded code is written in C, enabling the use of both \framac and DeepState.
SEGA will help us understand how our theory and tools can improve the correctness, reliability, and safety of cyber-physical and IoT applications.


\paragraph{Plan:}
%%%%%%%%% How will we carry out this case study?

First, we will model the protocol itself as a timed automata in \uppaal or \prism, in order to ensure that there is not a subtle flaw in the protocol itself, and to model our expectations of behavior in the real system.
Then, following our proposed workflow, we will automatically annotate the implementation with specifications extracted from % the specification of
the timed automata model and attempt to prove components of the code faithfully represent the intended behavior.
Either of these steps may expose the source of the mysterious networking failures.
Whether at this point the current problem is identified or not, we will finally use DeepState, driven by harnesses automatically generated by our tools, to generate tests of the implementation components in question.
Even if %(as we believe is unlikely)
\framac is able to prove most individual functions correct, which we believe is unlikely, the DeepState testing may expose faults that are not part of the specification.
% TRUONG: I commented out the following sentences because they are general (why we want to complement \framac with DeepState), therefore they should be included in the general discussion of the workflow.
%
%For instance, using libFuzzer with DeepState we can use LLVM's Undefined Behavior sanitized to catch some classes of undefined behavior that \framac does not take into account.  Furthermore, \framac's ability to prove properties about interactions of multiple functions operating in arbitrary sequence is often limited; such proofs are notoriously hard to construct in general.  DeepState allows us to hope to detect faults when we cannot prove correctness.  DeepState's ability to use symbolic execution as a back-end will be most useful for verifying single functions that are hard to verify with \framac, while state-of-the-art fuzzers will be most useful for sets of functions, or cases where symbolic execution fails to scale.
%
The above workflow will be conducted by an Embedded System Engineering student, who is familiar with the SEGA IoT system but does not have expertise in software verification and testing, using the software tools developed in this project.
Feedback from the engineer in this case study will inform us how to develop and improve the theory and tools for practical usages by non-expert users in real applications.


%%% Local Variables:
%%% mode: latex
%%% TeX-master: "main"
%%% End:


% \subsubsection{Case Study: Distributed Coordination in Multi-Robot Systems}
% \label{sec:case-study-robots}

% %%%%%%%%%%%%%%%%%%%% Intro to the application

\paragraph{Overview:}

Coordinated operation of multiple autonomous robots (multi-robot systems) has many important real-world applications \cite{multirobot2005,multirobotsurvey2013}.
For example, in a rescue, security, or disaster response mission, several autonomous aerial robots can coordinate to survey an area, monitor some target objects or activities, and guide ground robots or vehicles to complete complex tasks such as delivering aids.
In such applications, each robot is autonomous but has the capability to coordinate efficiently and safely with other robots to complete a shared mission, often in a distributed manner without any central coordinator.
Such distributed coordination is essential in real-world applications where the environment is constantly and unexpectly changing, but is also very challenging.
The Intelligent Control Systems (ICONS) Lab at NAU, directed by co-PI Nghiem, is developing distributed control and planning methods for multi-robot systems on a robotics platform that includes small quadcopters and four fully autonomous vehicles.
One of the most critical challenges of this research direction is the guarantee of the performance, in particular the safety, of a coordination plan, which is typically implemented in C code on the embedded computers of the robots and usually involves wireless inter-robot communication, sensing, and actuation.

\paragraph{Problem:}

Validation of a distributed coordination method for a multi-robot system is currently performed using a mix of theoretical proofs (for limited settings), extensive computer-based simulations, simulation-based falsification techniques, and real experiments.
Even when a method is validated by mathematical proofs and/or simulations, it often fails in real experiments due to discrepancies between models and real systems and between the method's design and its software implementation on the robots.
The methods and tools proposed in this project will help control and robotics researchers, who usually do not have expertise in software verification and testing, overcome this challenge.


\paragraph{Plan:}

First, we will model a coordination plan / algorithm for multiple robots as a potentially very complex network of timed automata.
Performance specifications will be expressed in temporal logics, e.g., the Signal Temporal Logic (STL) \cite{donze2010robust}, and checked against the model using verification and testing tools such as \uppaal or S-TaLiRo \cite{annpureddy2011s}.
This step ensures that the original coordination algorithm has no subtle flaws that lead to violations of the performance specifications.
An implementation of the algorithm in C code, distributed among the robots, will be developed by a robotics/control student.
The implementation will then be automatically annotated with the desired performance specifications by the tools developed in this project.
Next, we will attempt to prove, using our tools, that components of the code faithfully represent the coordination algorithm.
Given the complexity of multi-robot coordination algorithms, we do not expect \framac to be able to prove all the components of the implementation correct.
Consequently, we will use DeepState, driven by harnesses automatically generated by our tools, to generate tests of the implementation components not yet verified by \framac.
Finally, we will perform experiments of the coordination algorithm with aerial and ground robots in the ICONS Lab.
This case study will be conducted by a robotics/control graduate student in the ICONS Lab, using the software tools developed in this project.
Given the different nature and complexity of this application compared to the first case study on SEGA, the obtained feedback will be much valuable for the development and improvement of the proposed methods and tools for practical usages in a wide spectrum of real systems.


%%% Local Variables:
%%% mode: latex
%%% TeX-master: "main"
%%% End:


\subsection{Work Plan}
\begin{wrapfigure}[11]{r}{.35\textwidth}
%\begin{figure}[!tp]
%  \centering
  \resizebox{.35\textwidth}{!}{%
  \begin{ganttchart}[%Specs
    hgrid style/.style={black, dotted},
    vgrid, %={*2{black,dotted}, *1{black, dashed},
      %*2{black,dotted}, *1{black, dashed},
      %*2{black,dotted}, *1{black, dashed},
      %*2{black,dotted}, *1{black, solid}},
    x unit=3mm,
    y unit chart=5mm,
    y unit title=5mm,
    %time slot format=isodate,
    title height=1,
    milestone label font=\footnotesize,
    group label font=\bfseries\footnotesize,
    title label font=\bfseries\footnotesize,
    link/.style={->, thick},
    %bar/.style={fill=blue},
    %bar height=0.7,
    %group right shift=0,
    %group top shift=0.7,
    %group height=.3,
    %group peaks width={0.2},
    %inline
    ]{1}{36}
    % labels
    % \gantttitle{A two-years project}{24}\\  % title 1 
    \gantttitle[]{Year 1}{12}                 % title 1
    \gantttitle[]{Year 2}{12}
    \gantttitle[]{Year 3}{12} \\
    \gantttitle{Q1}{3}                      % title 3
    \gantttitle{Q2}{3}
    \gantttitle{Q3}{3}
    \gantttitle{Q4}{3}
    \gantttitle{Q1}{3}
    \gantttitle{Q2}{3}
    \gantttitle{Q3}{3}
    \gantttitle{Q4}{3}
    \gantttitle{Q1}{3}
    \gantttitle{Q2}{3}
    \gantttitle{Q3}{3} 
    \gantttitle{Q4}{3}\\    

    % \ganttgroup[inline=false]{Group 1}{1}{5}\\ 
    % \ganttbar[progress=10,inline=false]{Planning}{1}{4}\\
    % \ganttmilestone[inline=false]{Milestone 1}{9} \\

    % \ganttgroup[inline=false]{Group 2}{6}{12} \\ 
    % \ganttbar[progress=2,inline=false]{test1}{10}{19} \\
    % \ganttmilestone[inline=false]{Milestone 2}{17} \\
    % \ganttbar[progress=5,inline=false]{test2}{11}{20} \\
    % \ganttmilestone[inline=false]{Milestone 3}{22} \\       

    % \ganttgroup[inline=false]{Group 3}{13}{24} \\ 
    % \ganttbar[progress=90,inline=false]{Task A}{13}{15} \\ 
    % \ganttbar[progress=50,inline=false, bar progress label node/.append style={below left= 10pt and 7pt}]{Task B}{13}{24} \\ \\
    % \ganttbar[progress=30,inline=false]{Task C}{15}{16}\\ 
    % \ganttbar[progress=70,inline=false]{Task D}{18}{20} \\ 

    \ganttgroup[
        group/.append style={fill=blue}
    ]{WP1}{1}{36}\\ [grid]
    \ganttbluebar[
        name=T11
    ]{T1.1}{1}{12}\\ [grid]
    \ganttbluebar[
        name=T12
    ]{T1.2}{13}{36}\\ [grid]
    % \ganttlinkedbluebar{}{2014-10-7}{2014-10-10}
    % \ganttlinkedbluebar{}{2014-10-14}{2014-10-15}
    % \ganttlinkedbluebar{}{2014-10-17}{2014-10-17}
    % \ganttlinkedbluebar[name=FMEend]{}{2014-10-21}{2014-10-24}
    % \ganttlinkedbluebar{}{2014-10-28}{2014-10-31}\\ [grid]
    % \ganttbluebar[name=Manual]{Manual}{2014-10-30}{2014-10-31}
    % \ganttlinkedbluebar{}{2014-11-4}{2014-11-7} \ganttnewline[thick, black]

    \ganttgroup[
        group/.append style={fill=blue}
    ]{WP2}{1}{36}\\ [grid]
    \ganttbluebar[
        name=T21
    ]{T2.1}{1}{12}\\ [grid]
    \ganttbluebar[
        name=T22
    ]{T2.2}{13}{36}\\ [grid]

    \ganttgroup[
        group/.append style={fill=blue}
    ]{WP3}{1}{36}\\ [grid]
    \ganttbluebar[
        name=T31
    ]{T3.1}{1}{12}\\ [grid]
    \ganttbluebar[
        name=T32
    ]{T3.2}{13}{36}\\ [grid]
    \ganttbluebar[
        name=T41
    ]{T4.1}{7}{12}\\ [grid]        
    \ganttbluebar[
        name=T42
    ]{T4.2}{13}{36}\\ [grid]
    \ganttbluebar[
        name=T43
    ]{T4.3}{25}{36}    
    % %Implementing links
    % \ganttlink[link mid=0.75]{Documentation}{FME}
    % \ganttlink{FMETutorial}{FME}
  \end{ganttchart}}%
\caption{Project schedule.}%
\label{fig:project-schedule}%
% \end{figure}
%\vspace{-0.4in}
\end{wrapfigure}

The project will be organized into two phases, described by work
packages.  In the first phase, T3.1 will be conducted along with T1.1 and will inform the development in these tasks.
In the second phase, the focus will be on the application of tools in
T1.2 in tandem with T2.
Tasks related to case studies (tasks T4.1, and T4.2), whose results
and feedback will help refine the developed tools will be especially
emphasized in the final phases of the project.
The timeline of the tasks will be structured as shown in Figure~\ref{fig:project-schedule}.

\paragraph{Work Package 1 (WP1):}  This work package concerns the
development of and use of \acsl and \eacsl extensions for use in
embedded system implementation
code.


$\bullet$ T1.1: This task will consider needed extensions for handling
real-world embedded systems.  In particular, there will be a focus on
a study of the formal semantics of timed
automaton networks defined in \uppaal and \prism, to determine the
extent to which shared semantics can be assigned making it possible to
carry implementation annotations into such formal models (see T4.2)
and, in theory, translate properties from such models into implementation annotations.  As such, there will be close collaboration and iterative design steps between this task and the other work packages. %WP3 (Application).

$\bullet$ T1.2: This task will take feedback from applications of
tools to generate tests and proofs (T2) into account, to add annotations
that are focused on heuristic guidance for tools, not correctness per se.

One Ph.D. student will conduct this work, which will last for the entire duration of the project.

\paragraph{Evaluation:} Evaluation of
WP1 will be determined by ability of embedded engineers to agree that
the key properties, including those related to timed automata models, to be checked are (1) all representable by the
annotations (2) easy to construct (given the basic difficulty of
systems specification) (3) easy to read when produced by others and
(4) maintainable after introduction.

\paragraph{Work Package 2 (WP2):}  This work package covers
methods and tools to automatically translate \acsl/\eacsl-annotated code in
into a \deepstate test harness (Section~\ref{sec:framac2deepstate}),
the development of DeepState back-ends for CBMC and SPIN, with
appropriate mechanisms to ease the use of these tools, and
improvements to fuzzers to improve test generation:
\begin{itemize}[labelsep=3pt,leftmargin=12pt]
\item T2.1: This task will optimize the implementation of symbolic
  execution and fuzzing in DeepState, so that \acsl/\eacsl annotations
  and extensions from WP1 can be used effectively.
\item T2.2: This task will develop DeepState back-ends for CBMC and
  SPIN, inform annotations needed to handle loop bounds,
  memory tracking and matching, and make use of feedback from fuzzing.
\end{itemize}

The execution of this work package will also span the entire duration of the project.
Because the tasks in this package are also based on developing
verification and test generation tools (thus formal methods
expertise), the same Ph.D. student will work on WP1 and WP2.  We
separate the WPs primarily to emphasize that specification extensions
and tool support are somewhat orthogonal concerns, and evaluated differently.

\paragraph{Evaluation:} Evaluation of
WP2 will be determined by the application
of DeepState harnesses to generate tests for realistic
systems.  We will use benchmarks and simple examples to some
extent, but primarily rely on our connection to case studies.
In the case of test generation, in addition to faults
detected, we will use code coverage and other standard
benchmarks~\cite{FuzzerHicks}.  We expect to publish papers on
advances in fuzzing technology and
fundamental issues arising from the CBMC and SPIN back-ends with
respect to handling loop bounds and memory tracking/matching.

\paragraph{Work Package 3 (WP3):}  This work package will focus on the field applications described in Section~\ref{sec:case-study}, as both a way to inform the methodology and tool developments in the other work packages and case studies %in two completely different domains
to validate our methods and tools.
WP3 includes the following case studies:
\paragraph{Wireless sensor network (WSN) case studies on SEGA and DISCOVER.} This %application
is divided into two tasks:
\noindent  \begin{itemize}[labelsep=3pt,leftmargin=12pt]
\item T3.1: In this task, the %existing SEGA
  wireless sensor node systems will be studied thoroughly to extract
  the key requirements and characteristics of the embedded system
  implementations.  Timed automaton models of the communication
  protocol in each system, at different levels of abstraction, may be
  developed and formally verified in \uppaal and/or \prism, to inform
  task T1.1.  The system information and models resulting from this
  task will inform the semantics design and method developments in WP1
  and WP2.  As time allows we will extend this work to include sensing
  elements.
\item T3.2: This task will apply the tools developed in WP1 and WP2 to
  the WSN systems, %in order
  to detect and fix bugs in
  the %embedded software implementations of the
  communication protocol implementations; in particular, the bugs that
  cause the intermittent failures in
  SEGA. % mentioned in Section~\ref{sec:case-study}.
  It will also provide feedback to the other work packages to refine
  and improve our tools.
  \end{itemize}

\paragraph {Multi-robot system case study on DISCOVER.} This study is divided into three tasks:
\noindent \begin{itemize}[labelsep=3pt,leftmargin=12pt]
\item T4.1: In this task, a standard multi-robot coordination
  algorithm %currently used with our existing multi-robot system
  will be modeled as a network of timed automata.  Using our insights
  into the robotics application, we will express its performance
  specifications, particularly its safety requirements, in temporal
  logics and formally verify or test them in tools like \uppaal,
  \prism, or S-TaLiRo.  This task will extend the developed semantics
  and methods to applications beyond communication protocols, to
  identify further needed runtime extensions and semantic connections
  between timed automata theory, implementation annotations, and
  runtime checks.
\item T4.2: This task will apply the tools developed in WP1 and WP2,
  and the robot simulation environment of the DISCOVER platform, to
  the coordinated multi-robot system, in order to validate the
  implementation code and detect and fix possible bugs.  It will also
  provide feedback to the other work packages to refine and improve
  the tools developed in this project.
\item T4.3: This task will aim to use the work on the robotics effort
  to prototype a mapping from implementation code annotations in
  \acsl/\eacsl and extensions into skeletons of models in timed
  automata formalisms.
  \end{itemize}

As the tasks in this work package are conducted in tandem with WP1 and WP2, to form a feedback loop with the developments in other work packages, it will last for the entire duration of the project.
We expect that groups of undergraduate students, in collaboration with
an embedded systems Ph.D. student and the Ph.D. students in WP1 and WP2, will
perform the work.
Close collaboration with the DISCOVER team, led by Dr. Flikkema and Dr. Nghiem, is expected.

\paragraph{Evaluation:} In essence, this task \emph{is} the evaluation
aspect of our project, which forms one of the major thrusts of the
project.  The successful application of WP1 and WP2 tools to the case
studies is essentially the driving factor in determining our success
in the project.
%, and the key feedback to drive changes to our research
%priorities or technical choices.
The measure of success is: (1) faults detected and corrected; (2)
functionality proven correct using CBMC, symbolic execution engines,
or SPIN; (3) coverage and other measures of generated tests; and (4)
reported usability and value by embedded systems engineers,
particularly students.  For T4.3, the evaluation will be based on a
formal comparison of the extracted skeleton with full timed automata
models developed by embedded systems experts. The degree of success
will be estimated based on the correspondence with a real model.

%\subsubsection{Timeline}
%\label{sec:time-line}


%%% Local Variables:
%%% mode: latex
%%% TeX-master: "main"
%%% End:


\section{Contributions to Formal Methods and the Field}
\label{sec:contributions}
The contributions to formal methods proposed include:

\begin{itemize}
\item Fundamental contributions to integrating formal specification
languages developed for use in static analysis and theorem proving
with dynamic analysis, producing a common semantics for formal,
static, and dynamic checking of correctness.  Handling of timing and
interrupts are notable examples of problems to be addressed in this effort.
\item Enhanced ability of fuzzing and other test generation methods to
make use of information from formal specifications, and integrate
feedback about, e.g., specification coverage into test generation
heuristics.
\item Common semantics and a framework for fuzzing, symbolic execution, SAT/SMT-based
bounded model checking, and explicit-state model checking.
\item Approaches to using feedback from fuzzing to guide bounded or explicit-state model
checking.
\item Translations from implementation-level specification to
(probabilistic) timed automata models.
\end{itemize}

The contributions to the field include:

\begin{itemize}
\item New development and design methods that focus on
implementation-level specification of correctness of code as a guiding
principle for embedded and networked systems, addressing the major technical challenges of applying formal methods to practical embedded and networked systems.
\item Tactics and strategies for incorporating the above methods into
legacy efforts, where existing code bases require additional
specification and annotation.
\item Best-practices for using formal, static, and dynamic tools in
  debugging % legacy
  embedded and networked systems problems.
\end{itemize}

%\subsubsection{Evaluation Approach}
%\label{sec:eval}

Because of our focus on practical integration and extension of
existing tools and methods to real-world embedded systems,
our \emph{evaluation} of the degree to which these contributions have
been realized is described in the work plan above, integrated with
description of case study efforts.


%%% Local Variables:
%%% mode: latex
%%% TeX-master: "main"
%%% End:
